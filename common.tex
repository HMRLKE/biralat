\usepackage{ifthen}

\newcommand{\biralo}{Koszó Norbert}
\newcommand{\biraloKepesites}{okleveles villamosmérnök}
\newcommand{\jelolt}{Bódis-Szomorú András}
\newcommand{\kepzes}{villamosmérnök} %villamosmérnök vagy informatikus
\newcommand{\dolgozatcim}{Elektronikus terelők}
\newcommand{\dolgozattipus}{diplomaterv} %szakdolgozat vagy diplomaterv kis betűvel!
\newcommand{\erdemjegy}{jeles (5)}

%Néhány egyéb command, ne piszkáld!
\newcommand{\dolgozattipusahoz}{\ifthenelse{\equal{\dolgozattipus}{szakdolgozat}}{szakdolgozatához}{diplomatervéhez}}
\newcommand{\dolgozattipusat}{\ifthenelse{\equal{\dolgozattipus}{szakdolgozat}}{szakdolgozatát}{diplomatervét}}
\newcommand{\Dolgozattipus}{\ifthenelse{\equal{\dolgozattipus}{szakdolgozat}}{Szakdolgozat}{Diplomaterv}}
\newcommand{\alairas}{
\begin{flushleft}
\vspace*{1cm}
Budapest, \today
\end{flushleft}

\begin{flushright}
 \vspace*{1cm}
 \makebox[7cm]{\rule{6cm}{.4pt}}\\
 \makebox[7cm]{\emph{\biralo}}\\
 \makebox[7cm]{\biraloKepesites}
\end{flushright}
}



%További generátorok ^_^
\usepackage{arrayjob}

% felépítés 1-3
\newarray\felepitesopciok
\readarray{felepitesopciok}{Összességében elmondható, hogy a dolgozat felépítése sok kívánnivalót hagy maga után, amely nagyban nehezíti a befogadását.&Összességében elmondható, hogy a dolgozat jól felépített tematika szerint halad; amely nem, vagy csak kis mértékben akadályozza a befogadását.&Összességében elmondható, hogy a dolgozat kiválóan felépített tematika szerint halad, amely nagyban segíti a befogadását.}
\newcommand{\felepites}[1]{\felepitesopciok(#1)}

% stílus és tartalom 1-3
\newarray\stilusopciok
\readarray{stilusopciok}{A stílus és a tartalom nem felelnek meg a műfaji és szakmai követelményeknek.&A stílus és a tartalom többnyire megfelelnek a műfaji és szakmai követelményeknek.&A stílus és a tartalom egyaránt megfelelnek a műfaji és szakmai követelményeknek.}
\newcommand{\stilus}[1]{\stilusopciok(#1)}

% érthetőség 1-3
\newarray\erthetosegopciok
\readarray{erthetosegopciok}{A dolgozat nehezen érthető, a mondanivalót egyedi ábrák és/vagy kódrészletek nem, vagy nem megfelelő módon támasztják alá.&A dolgozat jellemzően érthető, a mondanivalót további egyedi ábrák és/vagy kódrészletek mégjobban alátámasztották volna.&A dolgozat jól érthető, a mondanivalót jól megválasztott egyedi ábrák és/vagy kódrészletek támasztják alá.}
\newcommand{\erthetoseg}[1]{\erthetosegopciok(#1)}

% korrektura 1-3
\newarray\korrekturaopciok
\readarray{korrekturaopciok}{A dolgozat sok formázási hibát és elírást tartalmaz, melyeken egy ismételt átolvasás segíthetett volna.&A dolgozat néhány formázási hibától eltekintve, többé-kevésbé mentes az elírásoktól, melyeken egy ismételt átolvasás segíthetett volna.&A dolgozat néhány kisebb formázási hibától eltekintve, gyakorlatilag mentes az elírásoktól.}
\newcommand{\korrektura}[1]{\korrekturaopciok(#1)}

% kérdések
\newcommand{\kerdes}[1]{\item #1}
\newcommand{\kerdesek}[1]{
    A dolgozattal kapcsolatban a következő kérdéseket tenném fel:
    \setlength{\parskip}{0pt}
    \begin{itemize}[noitemsep,nolistsep]
        #1
    \end{itemize}
    \setlength{\parskip}{\baselineskip}
}

\newarray\osszessegebenopciok
\readarray{osszessegebenopciok}{A dolgozatot összességében gyenge munkának tartom. A hallgató szakszerű, mérnöki hozzáállással nem analizálta a problémakört, a megoldás tárgyalása nincs logikusan felépítve és nem precíz.&A dolgozatot összességében jó munkának tartom. A hallgató szakszerű, mérnöki hozzáállással igyekezett analizálni a problémakört, a megoldás tárgyalása többnyire logikusan felépített és apróságokat leszámítva precíz, az alkalmazott technológiák megfelelően választottak.&A dolgozatot összességében kiváló munkának tartom. A hallgató szakszerű, mérnöki hozzáállással analizálta a problémakört, a megoldás tárgyalása logikusan felépített és precíz, az alkalmazott technológiák korszerűek.}
\newcommand{\osszessegeben}[1]{\osszessegebenopciok(#1)}

\newcommand{\elfogadas}[1]{\ifthenelse{\equal{#1}{igen}}{Összességében a szakdolgozat elfogadását javaslom.}{Összességében a szakdolgozat elfogadását nem javaslom.}}
