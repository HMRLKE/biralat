\documentclass[a4paper,12pt]{article}
\usepackage{t1enc}
\usepackage{ucs}
\usepackage[utf8x]{inputenc}
\def\magyarOptions{defaults=hu-min}
\usepackage[magyar]{babel}
\usepackage{amssymb}
\usepackage{anysize}
\paperwidth=21cm \paperheight29.7cm
\marginsize{2.5cm}{2.5cm}{2.5cm}{3cm} % {bal}{jobb}{felső}{alsó}
%\usepackage{pdflscape}
%\usepackage{subfig}
%\usepackage{sidecap}
\usepackage{listings}
%\usepackage{psfrag}
%\usepackage{graphicx}
%\usepackage{multicol}
\usepackage{times}

\lstset{
	basicstyle=\small\ttfamily, % print whole listing small
	keywordstyle=\color{black}\bfseries, % underlined bold black keywords
	identifierstyle=, 					% nothing happens
	commentstyle=\color{blue}, % white comments
	stringstyle=, 			% typewriter type for strings
	showstringspaces=false,     % no special string spaces
	aboveskip=3pt,
	belowskip=3pt,
%	columns=,
	backgroundcolor=,
	language=C,
} 		

\usepackage{ifthen}

\newcommand{\biralo}{Koszó Norbert}
\newcommand{\jelolt}{Bódis-Szomorú András}
\newcommand{\kepzes}{villamosmérnök} %villamosmérnök vagy informatikus
\newcommand{\dolgozatcim}{Elektronikus terelők}
\newcommand{\dolgozattipus}{diplomaterv} %szakdolgozat vagy diplomaterv kis betűvel!
\newcommand{\erdemjegy}{jeles (5)}

%Néhány egyéb command, ne piszkáld!
\newcommand{\dolgozattipusahoz}{\ifthenelse{\equal{\dolgozattipus}{szakdolgozat}}{szakdolgozatához}{diplomatervéhez}}
\newcommand{\dolgozattipusat}{\ifthenelse{\equal{\dolgozattipus}{szakdolgozat}}{szakdolgozatát}{diplomatervét}}
\newcommand{\Dolgozattipus}{\ifthenelse{\equal{\dolgozattipus}{szakdolgozat}}{Szakdolgozat}{Diplomaterv}}
\newcommand{\alairas}{
\begin{flushleft}
\vspace*{1cm}
Budapest, \today
\end{flushleft}

\begin{flushright}
 \vspace*{1cm}
 \makebox[7cm]{\rule{6cm}{.4pt}}\\
 \makebox[7cm]{\emph{\biralo}}\\
 \makebox[7cm]{okleveles villamosmérnök}
\end{flushright}
}

\title{\Dolgozattipus{} bírálat}
\author{\jelolt \\szigorló \kepzes{} részére}
\date{}
\begin{document}
\maketitle
% 1. A diplomaterv feladat tárgya, célkitűzése, időszerűsége.

% 2. A tartalom összefoglalása néhány mondatban.

% 3.A jelölt a kitűzött feladatokat megoldotta-e, a kidolgozás súlyozása megfelelt-e a kiírásnak? Esetleg milyen feltételeket nem vett figyelembe (egyáltalán nem, vagy nem kielégítően), vagy esetleg milyen pótlólagos feladatot oldott meg.

% 4. A kidolgozás értékelése
% 4.1 A kidolgozásnál alkalmazott gondolatmenet értékelése: logikus-e elfogadható-e stb.

% 4.2 Az alkalmazott számítási módszerek helyesek-e, azokat helyesen alkalmazta-e, a megfontolások helyesek-e?

% 4.3 A kidolgozásnál elkövetett esetleges hibák és azok minősítése.

% 4.4 A kapott eredmények értékelése (elfogatható-e, mondanak-e újat, közvetve vagy közvetlenül hasznosíthatók-e, stb.

% 5. Összefoglaló értékelés: a jelölt milyen képességekről, önállóságról, határozott ítélőképességről stb. tett tanúbizonyságot, valamint milyen mennyiségű munkát végzett?

% 6. A bírálat tartalmazzon javaslatot a diplomaterv elfogadására vagy elutasítására

\alairas

\end{document}
