\documentclass[a4paper,12pt]{article}
\usepackage{t1enc}
\usepackage{ucs}
\usepackage[utf8x]{inputenc}
\def\magyarOptions{defaults=hu-min}
\usepackage[magyar]{babel}
\usepackage{amssymb}
\usepackage{anysize}
\paperwidth=21cm \paperheight29.7cm
\marginsize{2.5cm}{2.5cm}{2.5cm}{2.5cm} % {bal}{jobb}{felső}{alsó}
%\usepackage{pdflscape}
%\usepackage{subfig}
%\usepackage{sidecap}
\usepackage{listings}
%\usepackage{psfrag}
%\usepackage{graphicx}
%\usepackage{multicol}
\usepackage{times}
\usepackage{enumitem}
\setlength{\parskip}{\baselineskip}

\lstset{
	basicstyle=\small\ttfamily, % print whole listing small
	keywordstyle=\color{black}\bfseries, % underlined bold black keywords
	identifierstyle=, 					% nothing happens
	commentstyle=\color{blue}, % white comments
	stringstyle=, 			% typewriter type for strings
	showstringspaces=false,     % no special string spaces
	aboveskip=3pt,
	belowskip=3pt,
%	columns=,
	backgroundcolor=,
	language=C,
}

\tolerance=1
\emergencystretch=\maxdimen
\hyphenpenalty=10000
\hbadness=10000

\usepackage{ifthen}

\newcommand{\biralo}{Koszó Norbert}
\newcommand{\jelolt}{Bódis-Szomorú András}
\newcommand{\kepzes}{villamosmérnök} %villamosmérnök vagy informatikus
\newcommand{\dolgozatcim}{Elektronikus terelők}
\newcommand{\dolgozattipus}{diplomaterv} %szakdolgozat vagy diplomaterv kis betűvel!
\newcommand{\erdemjegy}{jeles (5)}

%Néhány egyéb command, ne piszkáld!
\newcommand{\dolgozattipusahoz}{\ifthenelse{\equal{\dolgozattipus}{szakdolgozat}}{szakdolgozatához}{diplomatervéhez}}
\newcommand{\dolgozattipusat}{\ifthenelse{\equal{\dolgozattipus}{szakdolgozat}}{szakdolgozatát}{diplomatervét}}
\newcommand{\Dolgozattipus}{\ifthenelse{\equal{\dolgozattipus}{szakdolgozat}}{Szakdolgozat}{Diplomaterv}}
\newcommand{\alairas}{
\begin{flushleft}
\vspace*{1cm}
Budapest, \today
\end{flushleft}

\begin{flushright}
 \vspace*{1cm}
 \makebox[7cm]{\rule{6cm}{.4pt}}\\
 \makebox[7cm]{\emph{\biralo}}\\
 \makebox[7cm]{okleveles villamosmérnök}
\end{flushright}
}

\title{\Dolgozattipus{} bírálat \bigskip \\[0.2em]\large{\textbf{\jelolt} \\szigorló \kepzes{} hallgató \bigskip \\ \textbf{\dolgozatcim} \\ című dolgozatához}}
\author{}
\date{}

\begin{document}
\maketitle\vspace{-2cm}
% 1. A diplomaterv feladat tárgya, célkitűzése, időszerűsége.
A \dolgozattipus  ~tárgya és célkitűzései az élvonalbeli természetesnyelv-feldolgozás sajátjai. 2023-ban, amikor a generatív nyelvmodellek teljes iparágakat transzformálnak, kiváltkép értékes az a kompetencia, melyet a szigorló hallgató jelen munkája során - a dolgozat tanúbizonysága szerint - megszerzett.

% 2. A tartalom összefoglalása néhány mondatban.
A \dolgozattipus ~az absztraktív szövegkivonatolás generálási stratégiáit vizsgálja egy magyar nyelvű, és egy többnyelvű nagy nyelvmodellen alkalmazva. A kísérletek eredményei egyediek és érdekesek a szakmai közösség számára is. Az eredmények közérthetőségét azonban nagyban segítené egy, a kísérletek eredményeit összehasonlító táblázat, valamint az eremények kontextusba helyezése és kiértékelése.

% 3. Fejezetenként pár mondat azok tartalmáról / minőségéről / stb.
A bevezetés egy rövid kronológiai összefoglalóval kezdődik, amely tartalmazhatna több referenciát, vagy kevésbé határozott állításokat. A második fejezet a felhasznált szakirodalom egy szeletét hivatott bemutatni. Az alapfogalmak és eljárások bemutatása hiányos, de tekintettel a \dolgozattipus magas szintű mivoltára, akár elnézhető is lenne. Mindazonáltal, ha már van pl. "Backpropagation" al-al-alfejezet, akkor már illő bemutatni magát az algoritmust (nem csak a hibafüggvény számítási módját). Ha már megemlítésre kerül a SGD, akkor minimum hivatkozni kellene egy forrást; de leginkább egy-két mondattal bemutatni a lényegét és a működési elvét. A 2.1.1.6 alfejezetben található hiperparaméterek formális definíciója hiányzik. Ellenben bemutatásra kerülnek rekurrens neurális hálózatok is a 2.1.2 és a 2.1.3 alfejezetekben, amelyek nem kötődnek a szakdolgozat célkitűzéseihez, és nem is kerültek implementálásra a szokdolgozat tárgyát képző kísérletekben. Ezekben az alfejezetben is fellelhetőek olyan állítások, amelyek referenciát vagy legalább egy hiperhivatkozást kívánnának meg.

% 4.a Összefoglaló értékelés generált módon ^_^
\felepites{2} % 1-3
Mindazonáltal, a harmadik, egyik legfontosabb fejezet mindössze egy normaoldalnyi szöveget tartalmaz. Ez nem csak formai okokból aggályos, hanem tartalmi szempontból is. Nagyon keveset tudunk meg a tesztelt modellek tanításának körülményeiről. Már az is segített volna, ha a hallgató ide hivatkozza azt a cikket, amelyben publikálásra kerültek a modellek. Ha figyelembe vesszük, hogy a harmadik fejezet több mint felét kitevő HunSum-2 modellek nem képzik tárgyát a szakdolgozatnak, akkor a "Models" fejezet mindössze 9 sorból áll - minimális információ-tartalommal. Hiányzik továbbá a kiértékelő mérőszámok bemutatása és interpretálása (ROUGE, BLEU).
\stilus{2} % 1-3
A szakdolgozat rengeteg kísérletet mutat be - szóban. Az implementációs részletek teljesen hiányoznak a munkából. A forráskód hiánya pedig tovább nehezíti az egyes kísérletek megértését és bírálatát.
\erthetoseg{2} % 1-3
A szakdolgozat HunSum-1 ként hivatkozza az adatforrást (2.1.8 fejezet), és a kísérletek során vizsgált modelleket is (3.1 fejezet).
\korrektura{2} % 1-3
A dolgozat angol nyelven íródott, amely magával hozott nyelvtani és helyesírási hibákat is (Pl.: szórend, hiányzó alany, hiányzó vesszpk, stb.)




% 4.b Szabadszavas összefoglalás: A jelölt milyen képességekről, önállóságról, határozott ítélőképességről stb. tett tanúbizonyságot, valamint milyen mennyiségű munkát végzett?


% 4.c
%% Itt jönnek a kérdések ^_^
\kerdesek{
    \kerdes{A 2.1.8 sz. alfejezetben encoder-decoder modell került említésre. Feltehetően, ez a modell kerül Bert2Bert-ként feltüntetésre a dolgozat további részeiben. Hogyan használhatóak a BERT modellt alkotó enkóder blokkok dekódolásra? Mitől válnak dekóderré? Mutassa be a Bert2Bert architektúráját!}
    \kerdes{A 4.1 fejezetben (24. oldal, 2. sor) azt írja a hallgató, hogy több a tesztelésre kiválasztott cikk \emph{nem található meg} a HunSum-1 dataset-ben. Ebből arra lehet következtetni, hogy a többi tesztelésre kiválasztott cikk igen. A többi tesztelésre kiválasztott cikk a HunSum-1 tanító, validációs vagy tesztelésre szánt adatszeletéből származik? Milyen különbségekre lehet számítani egy teszt-adatponton valamint egy, a teszt adatbázisban nem megtalálható cikken történő inferencia között?}
}

% 5.a Végső összegzés ^_^
\osszessegeben{3} % 1-3

% 5.b A bírálat tartalmazzon javaslatot a diplomaterv elfogadására vagy elutasítására ^_^
\elfogadas{igen} % igen/nem

\alairas

\newpage
Néhány nyelvtani/helyesírási hiba:

\begin{itemize}

\item Page 1, Line 52:
Error: Incorrect use of "it's" (should be "its").
Corrected: "...NLP’s advancement through its history."

    \item Page 3, Abstract:
Original:  "It is important to have a competent model in order to reach acceptable results but using the right generational strategies is crucial in achieving the expected outputs."
Issue: Lack of comma before 'but'.

    \item Page 5, Introduction:

Original: "Natural Language Processing (NLP) has emerged as a groundbreaking field at the intersection of computer science artificial intelligence and linguistics fundamentally changing the way humans interact with machines and information."
Issue: Missing comma after 'computer science'.

    \item Page 10:

Original: "The 1970s and 1980s witnessed the emergence of statistical approaches to NLP led by advances in machine learning and data availability."
Issue: Missing comma after 'NLP'.
    \item Page 12:

Original: "The late 20th century and the early 21st century witnessed a big shift in NLP with the introduction of deep learning."
Issue: Missing comma after '21st century'.

%missing subject
    \item Page 10:
    
Original: "Throughout my thesis, will explore the methods mentioned above in the hope of
guiding the models toward generating desired outputs while minimizing unintended"
Issue: Missing subject 'I will explore'.
results.
%word order
    \item Page 25:
    
This to me signals that the model found no logical place for the
force word so it just put it at the end, making the summary the...
Issue: Word order.











\end{itemize}

\end{document}
